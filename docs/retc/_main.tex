% Options for packages loaded elsewhere
\PassOptionsToPackage{unicode}{hyperref}
\PassOptionsToPackage{hyphens}{url}
%
\documentclass[
]{article}
\usepackage{amsmath,amssymb}
\usepackage{lmodern}
\usepackage{iftex}
\ifPDFTeX
  \usepackage[T1]{fontenc}
  \usepackage[utf8]{inputenc}
  \usepackage{textcomp} % provide euro and other symbols
\else % if luatex or xetex
  \usepackage{unicode-math}
  \defaultfontfeatures{Scale=MatchLowercase}
  \defaultfontfeatures[\rmfamily]{Ligatures=TeX,Scale=1}
\fi
% Use upquote if available, for straight quotes in verbatim environments
\IfFileExists{upquote.sty}{\usepackage{upquote}}{}
\IfFileExists{microtype.sty}{% use microtype if available
  \usepackage[]{microtype}
  \UseMicrotypeSet[protrusion]{basicmath} % disable protrusion for tt fonts
}{}
\makeatletter
\@ifundefined{KOMAClassName}{% if non-KOMA class
  \IfFileExists{parskip.sty}{%
    \usepackage{parskip}
  }{% else
    \setlength{\parindent}{0pt}
    \setlength{\parskip}{6pt plus 2pt minus 1pt}}
}{% if KOMA class
  \KOMAoptions{parskip=half}}
\makeatother
\usepackage{xcolor}
\usepackage[margin=1in]{geometry}
\usepackage{color}
\usepackage{fancyvrb}
\newcommand{\VerbBar}{|}
\newcommand{\VERB}{\Verb[commandchars=\\\{\}]}
\DefineVerbatimEnvironment{Highlighting}{Verbatim}{commandchars=\\\{\}}
% Add ',fontsize=\small' for more characters per line
\usepackage{framed}
\definecolor{shadecolor}{RGB}{248,248,248}
\newenvironment{Shaded}{\begin{snugshade}}{\end{snugshade}}
\newcommand{\AlertTok}[1]{\textcolor[rgb]{0.94,0.16,0.16}{#1}}
\newcommand{\AnnotationTok}[1]{\textcolor[rgb]{0.56,0.35,0.01}{\textbf{\textit{#1}}}}
\newcommand{\AttributeTok}[1]{\textcolor[rgb]{0.77,0.63,0.00}{#1}}
\newcommand{\BaseNTok}[1]{\textcolor[rgb]{0.00,0.00,0.81}{#1}}
\newcommand{\BuiltInTok}[1]{#1}
\newcommand{\CharTok}[1]{\textcolor[rgb]{0.31,0.60,0.02}{#1}}
\newcommand{\CommentTok}[1]{\textcolor[rgb]{0.56,0.35,0.01}{\textit{#1}}}
\newcommand{\CommentVarTok}[1]{\textcolor[rgb]{0.56,0.35,0.01}{\textbf{\textit{#1}}}}
\newcommand{\ConstantTok}[1]{\textcolor[rgb]{0.00,0.00,0.00}{#1}}
\newcommand{\ControlFlowTok}[1]{\textcolor[rgb]{0.13,0.29,0.53}{\textbf{#1}}}
\newcommand{\DataTypeTok}[1]{\textcolor[rgb]{0.13,0.29,0.53}{#1}}
\newcommand{\DecValTok}[1]{\textcolor[rgb]{0.00,0.00,0.81}{#1}}
\newcommand{\DocumentationTok}[1]{\textcolor[rgb]{0.56,0.35,0.01}{\textbf{\textit{#1}}}}
\newcommand{\ErrorTok}[1]{\textcolor[rgb]{0.64,0.00,0.00}{\textbf{#1}}}
\newcommand{\ExtensionTok}[1]{#1}
\newcommand{\FloatTok}[1]{\textcolor[rgb]{0.00,0.00,0.81}{#1}}
\newcommand{\FunctionTok}[1]{\textcolor[rgb]{0.00,0.00,0.00}{#1}}
\newcommand{\ImportTok}[1]{#1}
\newcommand{\InformationTok}[1]{\textcolor[rgb]{0.56,0.35,0.01}{\textbf{\textit{#1}}}}
\newcommand{\KeywordTok}[1]{\textcolor[rgb]{0.13,0.29,0.53}{\textbf{#1}}}
\newcommand{\NormalTok}[1]{#1}
\newcommand{\OperatorTok}[1]{\textcolor[rgb]{0.81,0.36,0.00}{\textbf{#1}}}
\newcommand{\OtherTok}[1]{\textcolor[rgb]{0.56,0.35,0.01}{#1}}
\newcommand{\PreprocessorTok}[1]{\textcolor[rgb]{0.56,0.35,0.01}{\textit{#1}}}
\newcommand{\RegionMarkerTok}[1]{#1}
\newcommand{\SpecialCharTok}[1]{\textcolor[rgb]{0.00,0.00,0.00}{#1}}
\newcommand{\SpecialStringTok}[1]{\textcolor[rgb]{0.31,0.60,0.02}{#1}}
\newcommand{\StringTok}[1]{\textcolor[rgb]{0.31,0.60,0.02}{#1}}
\newcommand{\VariableTok}[1]{\textcolor[rgb]{0.00,0.00,0.00}{#1}}
\newcommand{\VerbatimStringTok}[1]{\textcolor[rgb]{0.31,0.60,0.02}{#1}}
\newcommand{\WarningTok}[1]{\textcolor[rgb]{0.56,0.35,0.01}{\textbf{\textit{#1}}}}
\usepackage{longtable,booktabs,array}
\usepackage{calc} % for calculating minipage widths
% Correct order of tables after \paragraph or \subparagraph
\usepackage{etoolbox}
\makeatletter
\patchcmd\longtable{\par}{\if@noskipsec\mbox{}\fi\par}{}{}
\makeatother
% Allow footnotes in longtable head/foot
\IfFileExists{footnotehyper.sty}{\usepackage{footnotehyper}}{\usepackage{footnote}}
\makesavenoteenv{longtable}
\usepackage{graphicx}
\makeatletter
\def\maxwidth{\ifdim\Gin@nat@width>\linewidth\linewidth\else\Gin@nat@width\fi}
\def\maxheight{\ifdim\Gin@nat@height>\textheight\textheight\else\Gin@nat@height\fi}
\makeatother
% Scale images if necessary, so that they will not overflow the page
% margins by default, and it is still possible to overwrite the defaults
% using explicit options in \includegraphics[width, height, ...]{}
\setkeys{Gin}{width=\maxwidth,height=\maxheight,keepaspectratio}
% Set default figure placement to htbp
\makeatletter
\def\fps@figure{htbp}
\makeatother
\setlength{\emergencystretch}{3em} % prevent overfull lines
\providecommand{\tightlist}{%
  \setlength{\itemsep}{0pt}\setlength{\parskip}{0pt}}
\setcounter{secnumdepth}{5}
\usepackage{booktabs}
\ifLuaTeX
  \usepackage{selnolig}  % disable illegal ligatures
\fi
\usepackage[]{natbib}
\bibliographystyle{plainnat}
\IfFileExists{bookmark.sty}{\usepackage{bookmark}}{\usepackage{hyperref}}
\IfFileExists{xurl.sty}{\usepackage{xurl}}{} % add URL line breaks if available
\urlstyle{same} % disable monospaced font for URLs
\hypersetup{
  pdftitle={Rのパッケージの活用},
  pdfauthor={Toshikazu Masumura},
  hidelinks,
  pdfcreator={LaTeX via pandoc}}

\title{Rのパッケージの活用}
\author{Toshikazu Masumura}
\date{2023-01-08}

\begin{document}
\maketitle

{
\setcounter{tocdepth}{2}
\tableofcontents
}
\hypertarget{ux306fux3058ux3081ux306b}{%
\section*{はじめに}\label{ux306fux3058ux3081ux306b}}
\addcontentsline{toc}{section}{はじめに}

\hypertarget{ggplot2}{%
\section{ggplot2の勧め}\label{ggplot2}}

\hypertarget{rux306eux4f5cux56f3ux74b0ux5883ux306eux6982ux8981}{%
\subsection{Rの作図環境の概要}\label{rux306eux4f5cux56f3ux74b0ux5883ux306eux6982ux8981}}

\begin{itemize}
\tightlist
\item
  base(graphics)
\item
  lattice
\item
  grid
\item
  ggplot2
\end{itemize}

\hypertarget{ggplot2ux3068ux306f}{%
\subsection{ggplot2とは}\label{ggplot2ux3068ux306f}}

ggplot2は,tidy dataにしておけば,使いやすい

\hypertarget{ggplot2ux306eux5229ux70b9}{%
\subsection{ggplot2の利点}\label{ggplot2ux306eux5229ux70b9}}

1つのデータをもとに,ちょっとの変更で棒グラフ,散布図,などなど各種のplotが可能
図が綺麗で,テーマの変更も簡単
facetによるグループ分けも便利

magrittrとの相性も良い.
特にファイル名を設定するときの\texttt{\%\$\%}や\texttt{\%T\%}など

ggplot2をサポートするライブラリも豊富

凡例の自動的な位置決めや配置など
ggpubrなども

\hypertarget{ggplot2ux306eux57faux672c}{%
\subsection{ggplot2の基本}\label{ggplot2ux306eux57faux672c}}

irisを例にするが,できれば,veganとかdaveのデータを使う
tidy dataへの変換が必要
コードのみか,詳しくは松村や比嘉の解説を参考に

aesthetics

geom\_point()
geom\_bar()
aes()
colour
group
size

\hypertarget{facetux3092ux4f7fux304aux3046}{%
\subsection{facetを使おう}\label{facetux3092ux4f7fux304aux3046}}

forループやsubset,あるいはdplyr::filterを使っていたものが,一気にできて便利
コードも簡単で見やすい

\hypertarget{ggsave}{%
\subsection{ggsave}\label{ggsave}}

\hypertarget{ux6587ux5b57ux5316ux3051ux3078ux306eux5bfeux51e6windows}{%
\subsection{文字化けへの対処(windows)}\label{ux6587ux5b57ux5316ux3051ux3078ux306eux5bfeux51e6windows}}

\hypertarget{themeux3092ux5c11ux3057ux3060ux3051ux8aacux660e}{%
\subsection{themeを少しだけ説明}\label{themeux3092ux5c11ux3057ux3060ux3051ux8aacux660e}}

\hypertarget{ux53c2ux8003ux66f8}{%
\subsection{参考書}\label{ux53c2ux8003ux66f8}}

\begin{itemize}
\tightlist
\item
  ggplot2
\item
  ggplot2のレシピ
\item
  unwin GDA
\item
  チートシート
\end{itemize}

\hypertarget{magrittr}{%
\section{magrritrの勧め}\label{magrittr}}

\hypertarget{tidyverseux3068magrittr}{%
\subsection{tidyverseとmagrittr}\label{tidyverseux3068magrittr}}

tidyverseは,Rでのデータ解析には欠かせないものになっている.
そこで,Rを起動時にtidyverseを読み込む人は多いだろう.
なお,tidyverseは1つのライブラリではなく,複数のライブラリからなる.

\begin{Shaded}
\begin{Highlighting}[]
\FunctionTok{library}\NormalTok{(tidyverse)}
\end{Highlighting}
\end{Shaded}

\begin{verbatim}
## -- Attaching packages --------------------------------------- tidyverse 1.3.1 --
\end{verbatim}

\begin{verbatim}
## v ggplot2 3.3.6      v purrr   0.3.4 
## v tibble  3.1.8      v dplyr   1.0.10
## v tidyr   1.2.1      v stringr 1.4.1 
## v readr   2.1.2      v forcats 0.5.1
\end{verbatim}

\begin{verbatim}
## -- Conflicts ------------------------------------------ tidyverse_conflicts() --
## x dplyr::filter() masks stats::filter()
## x dplyr::lag()    masks stats::lag()
\end{verbatim}

これらのライブラリの多く(forcats,tibble,stringr,dplyr,tidyr,purrr)で,\texttt{\%\textgreater{}\%} (パイプ)を使うことができる.
私は\texttt{\%\textgreater{}\%}がtidyverse独自のものだと勘違いをしていた.
しかし,\texttt{\%\textgreater{}\%}はもとはライブラリmagrittrの関数であり,そこからインポートされている.
そのため,tidyverseを読み込むと使うことができる.
\texttt{\%\textgreater{}\%}は,慣れるまでは何が便利なのか分からないが,慣れると欠かせなくなる.
さらに使っていると,癖なってしまって無駄にパイプを繋ぐこともある.
長過ぎるパイプは良くないのは当然であるものの,適度に使うとRでのプログラミングは非常に楽になる.

tidyverseの関数では,引数とするオブジェクトが統一されている.
具体的には,第1引数のオブジェクトがデータフレームやtibbleになっていることが多い.
そのため,パイプと相性が特に良い.

\hypertarget{ux3068ux305dux306eux4ef2ux9593}{%
\subsection{\texorpdfstring{\texttt{\%\textgreater{}\%}とその仲間}{\%\textgreater\%とその仲間}}\label{ux3068ux305dux306eux4ef2ux9593}}

\texttt{\%\textgreater{}\%}の仲間としては,以下の関数もある.

\begin{itemize}
\tightlist
\item
  \texttt{\%\textless{}\textgreater{}\%}
\item
  \texttt{\%T\textgreater{}\%}
\item
  \texttt{\%\$\%}
\end{itemize}

これらの関数は,tidyverseには含まれていないため,使用するにはmagrittrを読み込む必要がある.

\begin{Shaded}
\begin{Highlighting}[]
\FunctionTok{library}\NormalTok{(magrittr)}
\end{Highlighting}
\end{Shaded}

\begin{verbatim}
## 
## Attaching package: 'magrittr'
\end{verbatim}

\begin{verbatim}
## The following object is masked from 'package:purrr':
## 
##     set_names
\end{verbatim}

\begin{verbatim}
## The following object is masked from 'package:tidyr':
## 
##     extract
\end{verbatim}

\hypertarget{section}{%
\subsection{\texorpdfstring{\texttt{\%\textless{}\textgreater{}\%}}{\%\textless\textgreater\%}}\label{section}}

\hypertarget{ux4f7fux3044ux65b9}{%
\subsubsection{使い方}\label{ux4f7fux3044ux65b9}}

\texttt{\%\textless{}\textgreater{}\%}は,パイプを使って処理した内容を,最初のオブジェクトに再度代入するときに使う.
ほんの少しだけだが,コードを短くできる.

\begin{Shaded}
\begin{Highlighting}[]
\NormalTok{mpg }\CommentTok{\# 燃費データ}
\end{Highlighting}
\end{Shaded}

\begin{verbatim}
## # A tibble: 234 x 11
##    manufacturer model      displ  year   cyl trans drv     cty   hwy fl    class
##    <chr>        <chr>      <dbl> <int> <int> <chr> <chr> <int> <int> <chr> <chr>
##  1 audi         a4           1.8  1999     4 auto~ f        18    29 p     comp~
##  2 audi         a4           1.8  1999     4 manu~ f        21    29 p     comp~
##  3 audi         a4           2    2008     4 manu~ f        20    31 p     comp~
##  4 audi         a4           2    2008     4 auto~ f        21    30 p     comp~
##  5 audi         a4           2.8  1999     6 auto~ f        16    26 p     comp~
##  6 audi         a4           2.8  1999     6 manu~ f        18    26 p     comp~
##  7 audi         a4           3.1  2008     6 auto~ f        18    27 p     comp~
##  8 audi         a4 quattro   1.8  1999     4 manu~ 4        18    26 p     comp~
##  9 audi         a4 quattro   1.8  1999     4 auto~ 4        16    25 p     comp~
## 10 audi         a4 quattro   2    2008     4 manu~ 4        20    28 p     comp~
## # ... with 224 more rows
\end{verbatim}

\begin{Shaded}
\begin{Highlighting}[]
\NormalTok{tmp }\OtherTok{\textless{}{-}}\NormalTok{ mpg}
\NormalTok{tmp }\OtherTok{\textless{}{-}}
\NormalTok{  tmp }\SpecialCharTok{\%\textgreater{}\%}
\NormalTok{  dplyr}\SpecialCharTok{::}\FunctionTok{filter}\NormalTok{(year}\SpecialCharTok{==}\DecValTok{1999}\NormalTok{) }\SpecialCharTok{\%\textgreater{}\%}
\NormalTok{  tidyr}\SpecialCharTok{::}\FunctionTok{separate}\NormalTok{(trans, }\AttributeTok{into=}\FunctionTok{c}\NormalTok{(}\StringTok{"trans1"}\NormalTok{, }\StringTok{"trans2"}\NormalTok{, }\ConstantTok{NA}\NormalTok{)) }\SpecialCharTok{\%\textgreater{}\%}
  \FunctionTok{print}\NormalTok{()}
\end{Highlighting}
\end{Shaded}

\begin{verbatim}
## # A tibble: 117 x 12
##    manufac~1 model displ  year   cyl trans1 trans2 drv     cty   hwy fl    class
##    <chr>     <chr> <dbl> <int> <int> <chr>  <chr>  <chr> <int> <int> <chr> <chr>
##  1 audi      a4      1.8  1999     4 auto   l5     f        18    29 p     comp~
##  2 audi      a4      1.8  1999     4 manual m5     f        21    29 p     comp~
##  3 audi      a4      2.8  1999     6 auto   l5     f        16    26 p     comp~
##  4 audi      a4      2.8  1999     6 manual m5     f        18    26 p     comp~
##  5 audi      a4 q~   1.8  1999     4 manual m5     4        18    26 p     comp~
##  6 audi      a4 q~   1.8  1999     4 auto   l5     4        16    25 p     comp~
##  7 audi      a4 q~   2.8  1999     6 auto   l5     4        15    25 p     comp~
##  8 audi      a4 q~   2.8  1999     6 manual m5     4        17    25 p     comp~
##  9 audi      a6 q~   2.8  1999     6 auto   l5     4        15    24 p     mids~
## 10 chevrolet c150~   5.7  1999     8 auto   l4     r        13    17 r     suv  
## # ... with 107 more rows, and abbreviated variable name 1: manufacturer
\end{verbatim}

\begin{Shaded}
\begin{Highlighting}[]
\NormalTok{tmp }\OtherTok{\textless{}{-}}\NormalTok{ mpg}
\NormalTok{tmp }\SpecialCharTok{\%\textless{}\textgreater{}\%}
\NormalTok{  dplyr}\SpecialCharTok{::}\FunctionTok{filter}\NormalTok{(year}\SpecialCharTok{==}\DecValTok{1999}\NormalTok{) }\SpecialCharTok{\%\textgreater{}\%}
\NormalTok{  tidyr}\SpecialCharTok{::}\FunctionTok{separate}\NormalTok{(trans, }\AttributeTok{into=}\FunctionTok{c}\NormalTok{(}\StringTok{"trans1"}\NormalTok{, }\StringTok{"trans2"}\NormalTok{, }\ConstantTok{NA}\NormalTok{)) }\SpecialCharTok{\%\textgreater{}\%}
  \FunctionTok{print}\NormalTok{()}
\end{Highlighting}
\end{Shaded}

\begin{verbatim}
## # A tibble: 117 x 12
##    manufac~1 model displ  year   cyl trans1 trans2 drv     cty   hwy fl    class
##    <chr>     <chr> <dbl> <int> <int> <chr>  <chr>  <chr> <int> <int> <chr> <chr>
##  1 audi      a4      1.8  1999     4 auto   l5     f        18    29 p     comp~
##  2 audi      a4      1.8  1999     4 manual m5     f        21    29 p     comp~
##  3 audi      a4      2.8  1999     6 auto   l5     f        16    26 p     comp~
##  4 audi      a4      2.8  1999     6 manual m5     f        18    26 p     comp~
##  5 audi      a4 q~   1.8  1999     4 manual m5     4        18    26 p     comp~
##  6 audi      a4 q~   1.8  1999     4 auto   l5     4        16    25 p     comp~
##  7 audi      a4 q~   2.8  1999     6 auto   l5     4        15    25 p     comp~
##  8 audi      a4 q~   2.8  1999     6 manual m5     4        17    25 p     comp~
##  9 audi      a6 q~   2.8  1999     6 auto   l5     4        15    24 p     mids~
## 10 chevrolet c150~   5.7  1999     8 auto   l4     r        13    17 r     suv  
## # ... with 107 more rows, and abbreviated variable name 1: manufacturer
\end{verbatim}

\hypertarget{ux6ce8ux610fux70b9}{%
\subsubsection{注意点}\label{ux6ce8ux610fux70b9}}

試行錯誤でコードを書いている途中は,あまり使わないほうが良いだろう.
もとのオブジェクトが置き換わるので,処理結果が求めるものでないときに,もとに戻れないためである.

\hypertarget{t}{%
\subsection{\texorpdfstring{\texttt{\%T\textgreater{}\%}}{\%T\textgreater\%}}\label{t}}

\hypertarget{ux4f7fux3044ux65b9-1}{%
\subsubsection{使い方}\label{ux4f7fux3044ux65b9-1}}

処理途中に分岐をして別の処理をさせたいときに使う.

\begin{Shaded}
\begin{Highlighting}[]
\NormalTok{mpg }\CommentTok{\# 燃費データ}
\end{Highlighting}
\end{Shaded}

\begin{verbatim}
## # A tibble: 234 x 11
##    manufacturer model      displ  year   cyl trans drv     cty   hwy fl    class
##    <chr>        <chr>      <dbl> <int> <int> <chr> <chr> <int> <int> <chr> <chr>
##  1 audi         a4           1.8  1999     4 auto~ f        18    29 p     comp~
##  2 audi         a4           1.8  1999     4 manu~ f        21    29 p     comp~
##  3 audi         a4           2    2008     4 manu~ f        20    31 p     comp~
##  4 audi         a4           2    2008     4 auto~ f        21    30 p     comp~
##  5 audi         a4           2.8  1999     6 auto~ f        16    26 p     comp~
##  6 audi         a4           2.8  1999     6 manu~ f        18    26 p     comp~
##  7 audi         a4           3.1  2008     6 auto~ f        18    27 p     comp~
##  8 audi         a4 quattro   1.8  1999     4 manu~ 4        18    26 p     comp~
##  9 audi         a4 quattro   1.8  1999     4 auto~ 4        16    25 p     comp~
## 10 audi         a4 quattro   2    2008     4 manu~ 4        20    28 p     comp~
## # ... with 224 more rows
\end{verbatim}

\hypertarget{ux6ce8ux610fux70b9-1}{%
\subsubsection{注意点}\label{ux6ce8ux610fux70b9-1}}

分岐途中の結果をオブジェクトに代入するときには,\texttt{\textless{}-}ではなく,\texttt{\textless{}\textless{}-}を使用する.

\hypertarget{section-1}{%
\subsection{\texorpdfstring{\texttt{\%\$\%}}{\%\$\%}}\label{section-1}}

\hypertarget{ux4f7fux3044ux65b9-2}{%
\subsubsection{使い方}\label{ux4f7fux3044ux65b9-2}}

\hypertarget{ux6ce8ux610fux70b9-2}{%
\subsubsection{注意点}\label{ux6ce8ux610fux70b9-2}}

\end{document}
